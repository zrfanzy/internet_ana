% !TEX TS-program = pdflatex
% !TEX encoding = UTF-8 Unicode

% This is a simple template for a LaTeX document using the "article" class.
% See "book", "report", "letter" for other types of document.

\documentclass[11pt]{article} % use larger type; default would be 10pt

\usepackage[utf8]{inputenc} % set input encoding (not needed with XeLaTeX)
\usepackage{graphicx}
\usepackage{algorithm}
\usepackage{algorithmicx}
\usepackage{algpseudocode}
\usepackage{multirow}

%%% Examples of Article customizations
% These packages are optional, depending whether you want the features they provide.
% See the LaTeX Companion or other references for full information.

%%% PAGE DIMENSIONS
\usepackage{geometry} % to change the page dimensions
\geometry{a4paper} % or letterpaper (US) or a5paper or....
% \geometry{margin=2in} % for example, change the margins to 2 inches all round
% \geometry{landscape} % set up the page for landscape
%   read geometry.pdf for detailed page layout information
\usepackage{listings}
\usepackage{graphicx} % support the \includegraphics command and options

% \usepackage[parfill]{parskip} % Activate to begin paragraphs with an empty line rather than an indent

%%% PACKAGES
\usepackage{booktabs} % for much better looking tables
\usepackage{array} % for better arrays (eg matrices) in maths
\usepackage{paralist} % very flexible & customisable lists (eg. enumerate/itemize, etc.)
\usepackage{verbatim} % adds environment for commenting out blocks of text & for better verbatim
\usepackage{subfig} % make it possible to include more than one captioned figure/table in a single float
% These packages are all incorporated in the memoir class to one degree or another...

%%% HEADERS & FOOTERS
\usepackage{fancyhdr} % This should be set AFTER setting up the page geometry
\pagestyle{fancy} % options: empty , plain , fancy
\renewcommand{\headrulewidth}{0pt} % customise the layout...
\lhead{}\chead{}\rhead{}
\lfoot{}\cfoot{\thepage}\rfoot{}

%%% SECTION TITLE APPEARANCE
\usepackage{sectsty}
\allsectionsfont{\sffamily\mdseries\upshape} % (See the fntguide.pdf for font help)
% (This matches ConTeXt defaults)

%%% ToC (table of contents) APPEARANCE
\usepackage[nottoc,notlof,notlot]{tocbibind} % Put the bibliography in the ToC
\usepackage[titles,subfigure]{tocloft} % Alter the style of the Table of Contents
\renewcommand{\cftsecfont}{\rmfamily\mdseries\upshape}
\renewcommand{\cftsecpagefont}{\rmfamily\mdseries\upshape} % No bold!

%%% END Article customizations

%%% The "real" document content comes below...

\title{Lab 8: PageRank}
\author{Ruofan Zhou}
%\date{} % Activate to display a given date or no date (if empty),
         % otherwise the current date is printed 

\begin{document}
\maketitle

\section{Random Surfer Model}
\subsection{NaiveRandomSurfer}
The method \textbf{compute} is complemented as below:\\
\begin{lstlisting}[language=JAVA]
public PageRank compute(Graph graph) {
        int nbNodes = graph.size();
        double[] probabilities = new double[nbNodes];
        
        Random rand = new Random();
        int nownode = rand.nextInt(nbNodes);
        for (int loops = 0; loops < NB_ITERATIONS; ++loops) {
        		List<Integer> nei = graph.neighbors(nownode);
        		int mount = nei.size();
      	  	int next = rand.nextInt(mount);
        		next = (Integer) nei.get(next);
        		probabilities[next] += 1;
   	     	nownode = next;
        }
        for (int i = 0; i < nbNodes; ++i)
        		probabilities[i] = (double)probabilities[i]
			/ NB_ITERATIONS;
        return new PageRank(graph, probabilities);
    }
\end{lstlisting}
\\
The \textbf{component.graph} is not a connected graph, thus the NaiveRandomSurfer can't go to somewhere of the graph(i.e., some nodes' PageRank is 0).\\
The \textbf{absorbing.graph} has a node with no outdegree, thus cause an Exception when using NaiveRandomSurfer.\\

\subsection{RandomSurfer}
The method \textbf{compute} is complemented as below:\\
\begin{lstlisting}[language=JAVA]
public PageRank compute(Graph graph) {
        int nbNodes = graph.size();
        double[] probabilities = new double[nbNodes];

        Random rand = new Random();
        int nownode = rand.nextInt(nbNodes);
        for (int loops = 0; loops < NB_ITERATIONS; ++loops) {
        		List<Integer> nei = graph.neighbors(nownode);
        		int mount = nei.size();
        		double rant = rand.nextDouble();
        		if (rant - 0.15 < 1e-9 || mount < 1) {
        			nownode = rand.nextInt(nbNodes);
        			probabilities[nownode] += 1;
        		}
        	else {
        		int next = rand.nextInt(mount);
        		next = (Integer) nei.get(next);
        		probabilities[next] += 1;
        		nownode = next;
        	}
        }
        for (int i = 0; i < nbNodes; ++i)
        		probabilities[i] = (double)probabilities[i]
			/ NB_ITERATIONS;
        return new PageRank(graph, probabilities);
    }
\end{lstlisting}
\\
It now can deal with the component.graph and absorbing.graph.

\section{Power Iteration Method}

\subsection{googleMatrix}

\begin{lstlisting}[language=JAVA]
public static double[][] googleMatrix(Graph graph) {
        int n = graph.size();
        double[][] probs = new double[n][n];
        double delta = DAMPING_FACTOR / n;
        double theta = 1 - DAMPING_FACTOR;

        double[][] H = new double[n][n];
        for (int i = 0; i < n; ++i) {
        		List<Integer> nei = graph.neighbors(i);
        		if (nei.size() == 0) {
        			for (int j = 0; j < n; ++j) {
        				H[i][j] = 1 / (double)n;
        			}
        	} else {
        		for (int j = 0; j < n; ++j) {
        			if ((i != j) && graph.containsEdge(i, j)) {
        				H[i][j] = 1 / (double)nei.size();
        			}
        		}
        	}
        } // ^H
        for (int i = 0; i < n; ++i)
        		for (int j = 0; j < n; ++j)
        			probs[i][j] = theta * H[i][j] +
			(1 - theta) / (double) n;
        
        return probs;
    }
\end{lstlisting}
\subsection{20 First PageRank}
1    United States (PR: 0.73570\%)\\
2    United Kingdom (PR: 0.52027\%)\\
3    France (PR: 0.48089\%)\\
4    Europe (PR: 0.43404\%)\\
5    England (PR: 0.39715\%)\\
6    Germany (PR: 0.38127\%)\\
7    Latin (PR: 0.37843\%)\\
8    World War II (PR: 0.35895\%)\\
9    India (PR: 0.34712\%)\\
10   English language (PR: 0.33081\%)\\
11   Australia (PR: 0.32051\%)\\
12   London (PR: 0.30620\%)\\
13   Japan (PR: 0.30054\%)\\
14   Italy (PR: 0.28641\%)\\
15   Canada (PR: 0.28597\%)\\
16   Water (PR: 0.28466\%)\\
17   China (PR: 0.28365\%)\\
18   Spain (PR: 0.24892\%)\\
19   Russia (PR: 0.24787\%)\\
20   Animal (PR: 0.24766\%)\\

\subsection{Gaming The System}

\subsection{AddIncomingEdges}

\begin{lstlisting}[language=JAVA]
public static void addIncomingEdges(Graph graph, int node) {
    	PageRankAlgorithm algo = new PowerMethod();
        PageRank pr = algo.compute(graph);
        int max = -1;
        double maxans = -1;
        for (int loops = 0; loops < 300; ++loops) {
        		max = -1;
        		for (int i = 0; i < graph.size(); ++i) {
    			if (i == node) continue;
            		if (!graph.containsEdge(i, node)) {
            			double now = pr.get(i) /
					(double)graph.neighbors(i).size();
            			if (max == -1) {max = i; maxans = now;}
            		else if (maxans - now < 1e-9) {
            			max = i;
            			maxans = now;
            		}
            	}
        	}
        	if (max < 0) return;
    		graph.addEdge(max, node);
        }
    }
\end{lstlisting}
\\And ran 98.51\% of the standard best pagerank. \\
\subsection{AddEdges}
I tried some algorthms but none of them work better than \textbf{addIncomingEdges}.
\end{document}
