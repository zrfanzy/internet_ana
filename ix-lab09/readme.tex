% !TEX TS-program = pdflatex
% !TEX encoding = UTF-8 Unicode

% This is a simple template for a LaTeX document using the "article" class.
% See "book", "report", "letter" for other types of document.

\documentclass[11pt]{article} % use larger type; default would be 10pt

\usepackage[utf8]{inputenc} % set input encoding (not needed with XeLaTeX)
\usepackage{graphicx}
\usepackage{algorithm}
\usepackage{algorithmicx}
\usepackage{algpseudocode}
\usepackage{multirow}

%%% Examples of Article customizations
% These packages are optional, depending whether you want the features they provide.
% See the LaTeX Companion or other references for full information.

%%% PAGE DIMENSIONS
\usepackage{geometry} % to change the page dimensions
\geometry{a4paper} % or letterpaper (US) or a5paper or....
% \geometry{margin=2in} % for example, change the margins to 2 inches all round
% \geometry{landscape} % set up the page for landscape
%   read geometry.pdf for detailed page layout information
\usepackage{listings}
\usepackage{graphicx} % support the \includegraphics command and options

% \usepackage[parfill]{parskip} % Activate to begin paragraphs with an empty line rather than an indent

%%% PACKAGES
\usepackage{booktabs} % for much better looking tables
\usepackage{array} % for better arrays (eg matrices) in maths
\usepackage{paralist} % very flexible & customisable lists (eg. enumerate/itemize, etc.)
\usepackage{verbatim} % adds environment for commenting out blocks of text & for better verbatim
\usepackage{subfig} % make it possible to include more than one captioned figure/table in a single float
% These packages are all incorporated in the memoir class to one degree or another...

%%% HEADERS & FOOTERS
\usepackage{fancyhdr} % This should be set AFTER setting up the page geometry
\pagestyle{fancy} % options: empty , plain , fancy
\renewcommand{\headrulewidth}{0pt} % customise the layout...
\lhead{}\chead{}\rhead{}
\lfoot{}\cfoot{\thepage}\rfoot{}

%%% SECTION TITLE APPEARANCE
\usepackage{sectsty}
\allsectionsfont{\sffamily\mdseries\upshape} % (See the fntguide.pdf for font help)
% (This matches ConTeXt defaults)

%%% ToC (table of contents) APPEARANCE
\usepackage[nottoc,notlof,notlot]{tocbibind} % Put the bibliography in the ToC
\usepackage[titles,subfigure]{tocloft} % Alter the style of the Table of Contents
\renewcommand{\cftsecfont}{\rmfamily\mdseries\upshape}
\renewcommand{\cftsecpagefont}{\rmfamily\mdseries\upshape} % No bold!

%%% END Article customizations

%%% The "real" document content comes below...

\title{Lab 9: Flajolet-Martin}
\author{Ruofan Zhou}
%\date{} % Activate to display a given date or no date (if empty),
         % otherwise the current date is printed 

\begin{document}
\maketitle

\section{Questions}
\subsection{Q1 Compare the estimates you get using the average and the median. Both have advantages and disadvantages—describe what you observe, and explain.}

The average has more statistical significance but the estimates seems absurd because small value of R can't have much weight when using average of $2^{R}$ so the values are fully discrete. While for the median, the values are more stable but what I can got is only power of 2, it should be more statistical.

\subsection{Q2 Include your final version of this function.}

\begin{lstlisting}[language=JAVA]
public long countUniqueWords(Iterator<String> iter) {
    while (iter.hasNext()) {
        this.processWord(iter.next());
    }
    double[] arr = new double[(hashFunctions.length + 2) / 3];
    for (int i = 0; i < hashFunctions.length; ++i) {
	arr[i / 3] = 2 << maxZeros[i];
        	int sum = 1;
        	++ i;
        	if (i < hashFunctions.length) {
        		sum ++;
        		arr[i / 3] += 2 << maxZeros[i];
        		++i;
        		if (i < hashFunctions.length) {
        			sum ++;
        			arr[i / 3] += 2 << maxZeros[i];
        			++i;
        		}
        	}
        	arr[i / 3] = arr[i / 3] / ((double)sum * 1.0);
        }
        return (long)median(arr);
    }
\end{lstlisting}

\subsection{Q3 What estimate Nˆ do you get? Look at the value of nbBits in the function main(), and explain what problem might arise in this case.}

I got \textbf{33554432}, and it is exactly \textbf{$2^{24}$}(which 24 is the nbBits in the function main()), so the problem is, the answer is much larger than \textbf{$2^{24}$} while we just can get number \textbf{$2^{24}$} because it is limited by nbBits.

\subsection{Q4 What estimate Nˆ do you get this time? Based on this estimate, what is the smallest value that nbBits could take for this particular instance of dataset?}

I got \textbf{268435456}. Based on this estimate, the smallest value of nbBits is {log(268435456)}, is \textbf{28}. To avoid overflow, I think the smallest value we set should be \textbf{30}.

\subsection{Q5 How much memory (in bytes) would you roughly need if you if you were to run ExactCount on the dataset? Justify your answer.}
The space complexity is O(nlogn), while n is about \textbf{268435456}. The words's average length is about \textbf{7}, so for the data, a string covers about \textbf{7} bytes. Thus the extimate memory is about $268435456\times log(268435456)\times7$, is \textbf{$5\times10^{10}$} bytes.
\end{document}
