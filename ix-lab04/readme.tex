% !TEX TS-program = pdflatex
% !TEX encoding = UTF-8 Unicode

% This is a simple template for a LaTeX document using the "article" class.
% See "book", "report", "letter" for other types of document.

\documentclass[11pt]{article} % use larger type; default would be 10pt

\usepackage[utf8]{inputenc} % set input encoding (not needed with XeLaTeX)
\usepackage{graphicx}
\usepackage{algorithm}
\usepackage{algorithmicx}
\usepackage{algpseudocode}
\usepackage{multirow}

%%% Examples of Article customizations
% These packages are optional, depending whether you want the features they provide.
% See the LaTeX Companion or other references for full information.

%%% PAGE DIMENSIONS
\usepackage{geometry} % to change the page dimensions
\geometry{a4paper} % or letterpaper (US) or a5paper or....
% \geometry{margin=2in} % for example, change the margins to 2 inches all round
% \geometry{landscape} % set up the page for landscape
%   read geometry.pdf for detailed page layout information

\usepackage{graphicx} % support the \includegraphics command and options

% \usepackage[parfill]{parskip} % Activate to begin paragraphs with an empty line rather than an indent

%%% PACKAGES
\usepackage{booktabs} % for much better looking tables
\usepackage{array} % for better arrays (eg matrices) in maths
\usepackage{paralist} % very flexible & customisable lists (eg. enumerate/itemize, etc.)
\usepackage{verbatim} % adds environment for commenting out blocks of text & for better verbatim
\usepackage{subfig} % make it possible to include more than one captioned figure/table in a single float
% These packages are all incorporated in the memoir class to one degree or another...

%%% HEADERS & FOOTERS
\usepackage{fancyhdr} % This should be set AFTER setting up the page geometry
\pagestyle{fancy} % options: empty , plain , fancy
\renewcommand{\headrulewidth}{0pt} % customise the layout...
\lhead{}\chead{}\rhead{}
\lfoot{}\cfoot{\thepage}\rfoot{}

%%% SECTION TITLE APPEARANCE
\usepackage{sectsty}
\allsectionsfont{\sffamily\mdseries\upshape} % (See the fntguide.pdf for font help)
% (This matches ConTeXt defaults)

%%% ToC (table of contents) APPEARANCE
\usepackage[nottoc,notlof,notlot]{tocbibind} % Put the bibliography in the ToC
\usepackage[titles,subfigure]{tocloft} % Alter the style of the Table of Contents
\renewcommand{\cftsecfont}{\rmfamily\mdseries\upshape}
\renewcommand{\cftsecpagefont}{\rmfamily\mdseries\upshape} % No bold!

%%% END Article customizations

%%% The "real" document content comes below...

\title{Lab 4: Dimensionality Reduction}
\author{Ruofan Zhou}
%\date{} % Activate to display a given date or no date (if empty),
         % otherwise the current date is printed 

\begin{document}
\maketitle

\section{Faces}
1. Done\\
2.  The eignvalue plot is as below:\\
\centerline{
\includegraphics[width=13cm]{pic/p4}}\\
3. The first principal component exlain \emph{0.378182} of the variance.\\
The first two principal components exlain \emph{0.458936} of the variance.
The first three principal components exlain \emph{0.505655} of the variance.
4. I sorted the variances and get the plot: \\
\centerline{
\includegraphics[width=13cm]{pic/p2}}\\
but still can't really find how many dimensions it can be precised to.\\
5. It's really hard and I give it up.\\

\subsection{Principal Component Analysis}
6. Done.\\
7. See the plot of the variances as below. It's obvious that \emph{7} components are stand out, and it's equal to the answer of quesion 2.\\
\centerline{
\includegraphics[width=10cm]{pic/p3}}
8. \& 9. Done. \\
10. With very happy(mouth up) face[dim 1], whose eyes are slightly larger than normal[dim 2] and green[dim 3]. His hair is relatively[dim 4] and is in bright yellow color[dim 5]. Normal face color(not too dark or too bright)[dim 6], and pobably with a brown hat[dim 7].\\

\section{Smartvote}
1. There are \emph{5000} faces and \emph{50} measurements.\\
2.  See the \emph{variance-dimension} plot as below:\\
\centerline{
\includegraphics[width=13cm]{pic/p1}}\\
3. I perfer to describe the face with \emph{7} variable parts: \emph{hat color, hair length, hair color, eye size, face color, mouth shape, eye color}. \\
4. I sorted the variances and get the plot: \\
\centerline{
\includegraphics[width=13cm]{pic/p2}}\\
but still can't really find how many dimensions it can be precised to.\\
5. It's really hard and I give it up.\\

\subsection{Principal Component Analysis}
6. Done.\\
7. See the plot of the variances as below. It's obvious that \emph{7} components are stand out, and it's equal to the answer of quesion 2.\\
\centerline{
\includegraphics[width=10cm]{pic/p3}}
8. \& 9. Done. \\
10. With very happy(mouth up) face[dim 1], whose eyes are slightly larger than normal[dim 2] and green[dim 3]. His hair is relatively[dim 4] and is in bright yellow color[dim 5]. Normal face color(not too dark or too bright)[dim 6], and pobably with a brown hat[dim 7].\\
\end{document}
